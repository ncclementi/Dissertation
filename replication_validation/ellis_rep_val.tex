% !TEX root = ../thesis_main.tex

\section{Validation of \pygbe and Replication of Ellis et al. 2016  et al 2005} \label{chap:rep_val_ellis}
\graphicspath{{replication_validation/figs/}}

The work of Ellis et al. "Aspect-ratio driven evolution of high-order resonant modes and near-field distributions in
localized surface phonon polariton nanostructures." \cite{ellis2016} has both computational and experimental results, what makes 
it a perfect candidate to perform a validation as well as a replication study. Ellis and coworkers study the excitation of 
multipolar localized surface polaritons (SPhP), by computing and measuring the polarized reflectance on 4H-SiC pillars of
fixed width ($W = 400$ nm), fixed hight ($H=950$ nm) and varied length ($L=400-4800$ nm). To reduce coupling these pillars are 
pattern on a square grid with $P = L + 500$ nm. In their experiments (simulations), they measure (compute) the polarized reflectance
where the incident polarization is oriented parallel or perpendicular to the length ($L$) of the pillars.

We started by replicating a computational result shown in Figure S4 of their supplementary material. Figure S4 of the 
supplementary material shows simulation results for the resonance spectral position of the lower frequency mode when the angle of 
incidence is 22 degrees and the polarization is parallel to the length of the pillars. They present results for separations of $500$ nm 
(red curve) and $5000$ (black curve), being the latter a good candidate for replication with \pygbe since a larger gap diminishes 
the coupling (not included in our model). The setup in our computations consists of a single pillar with no substrate.
Secondly, we aimed to replicate the results of Figure 2a of the main paper, 
corresponding to reflectance measurements across the wave number for pillars of aspect ratio $AR=4$, angle of incidence 22 degrees, 
and incoming parallel polarization. Since for this the authors also reported experimental results we decided to us them for the 
validation of our solver.

\textbf{Differences in method and input data}

\begin{itemize}

\item {The simulations of Ellis et al. compute the solution of the Maxwell's equations using the RF package
of the finite element solver in the commercial software COMSOL. Their setup consists of one pillar over a 
substrate, with periodic boundary conditions to emulate the array of pillar used in their experiments. In our 
solver we use a boundary element method in the quasistatic approximation, which is suitable since the wavelengths
involved are in the range $10000-12500$ nm, and are considerably larger than the pillar's dimensions. We compute 
the extinction cross section, where the resonance express as peaks instead of dips as in the reflection plots of 
Ellis et al. The intensity of the peaks is not comparable, however, we are looking to match the wave number at which
they happen}



\end{itemize}
