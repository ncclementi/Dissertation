% !TEX root = ../thesis_main.tex

\chapter{Replicating Rockstuhl et al 2005} \label{chap:rep_rockstuhl}
\graphicspath{{replication_validation/figs/}}

When looking for results to attempt the validation of \pygbe we found the study of Rockstuhl et al. 2005 \cite{rockstuhl2005}. 
Even though their work consist on simulations and therefore does not qualify for a validation study, we decided to 
attempt the replication of one of their results. 

Rockstuhl and coworkers, in their paper "Analysis of the phonon-polariton response of silicon carbide microparticles 
and nanoparticles by use of the boundary element method", study the phonon-polariton response of silicon carbide (SiC)
nanoparticles using boundary element method. They use a two dimensional model developed previously on their group \cite{rockstuhl2003}
to analyze 6H-SiC multiple "cylindrical particles" (third dimension tends to infinity). The results presented on Figure 14 of their paper  
present the scattering cross-section of a SiC rectangular cylinder for different aspect ratios, and the case with $a = 672$ nm and $b = 328$ nm
was a perfect candidate given that these dimensions comply with our quasistatic approach.

In the work of Rockstuhl et al, they compute scattering cross section while with \pygbe we compute the extinction cross-section 
(absorption plus scattering). In the quasistatic regime absorption dominates over scattering, therefore these results are not directly
comparable. However, when having materials which show narrow and sharp peaks on their spectra, the wavelength at which the peaks occur 
in the scattering and extinction spectra, are nearly the same. For example, we can see ion the work of Wiley et al. \cite{wiley-etal-2006}
that for a silver nanocube the main peaks for extinction and scattering are aligned. In the case of Silicon carbide we are talking about a 
material which figure of merit (this correlate to narrow peak and sharp ones) is even better than for silver, so in our case we are confident that
the peaks on scattering and extinction are aligned, therefore we can compare with Rockstuhl results.

Differences between simulations

Method and model
\begin{itemize}

\item {The main difference between the simulations in Rockstuhl and the ones performed with \pygbe is that the original results where obtained 
using a 2D boundary element method that solves full Maxwell equations, while we solve a 3D problem with the quasistatic limit approximation.}


\item{ Rockstuhl et al. work did not have a section with details on their simulations such as discretization of the geometries or parameters 
involved in their simulations. We chose parameters for our solver based on our previous experiences, and for the discretization of our 
geometries we performed a grid-independence study to ensure that we are minimizing discretization errors. It is worth noting that Rockstuhl
et al. geometries consist in two dimensions (infinite third dimension), while we perform the computations using a full 3D representation of the geometry.} 

\item {Regarding the complex dielectric constant, the study we aim to replicate uses 6h-SiC and their data comes from a source that we were not able
to replicate. As a replacement, we are using experimental data of 4H-SiC provided to us by the authors of Ellis et al. \cite{ellis2016} via 
private communications.}

\item {We compute extinction cross-section while they compute scattering cross-section, but since the quantity of interest is the wavelength
at which the resonance peaks occur we can compare despite this difference.}

\end{itemize}