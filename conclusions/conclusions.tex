% !TEX root = ../thesis_main.tex
\chapter{Conclusions}

In this work, we extended the implicit-solvent model of electrostatics implemented in \pygbe \cite{CooperClementiBarba2015}
to study nanoplasmonics in the quasistatic limit for applications in biosensing \cite{ClementiETal2017, ClementiETal2019}.
We applied this model to simulations of localized surface plasmon resonance, where we computed the extinction cross section of 
conductive nanoparticles and computed the shift in the resonance frequency when having biomolecules in the vicinity of the 
nanoparticle.

\textbf{Contributions of this work}

We developed an open source boundary integral solver for computational nanoplasmonics in the quasistatic limit. We extended \pygbe to work 
with complex values, and added the functionality needed to solve localized surface plasmon resonance problems. \pygbe 
benefits from an algorithmic acceleration with a treecode, and hardware acceleration with GPUs, making this solver 
suitable to compute problems of at least a half million boundary elements, which is required to represent the solvent excluded 
surface of the biomolecule accurately.

We verified our solver against an analytical solution based on Mie theory for silver nanospheres (section \ref{sec:verification}), and 
presented convergence studies for our results, building confidence on the suitability of our boundary integral model and the  
correctness of our solver. We performed a sensitivity study of an LSPR biosensor model by computing the resonance-frequency shift
while varying the distance between the nanoparticle and the biomolecules. We showed that our boundary element method approach in
the quasistatic limit can capture the characteristic behavior of LSPR biosensors (section \ref{sec:lspr_response_bsa}).

We explored different reduced-order models for the protein, and showed that geometries like ellipsoids and spheres
are not accurate models for the protein representation compared to using a mesh constructed from its crystal structure. Even though
none of these models were accurate compared to the crystal structure model of the protein, the volume-equivalent ellipsoidal model performed better 
compared to the surface-equivalent ellipsoidal model and the volume-equivalent spherical model. Using volume-equivalent ellipsoids for the protein representation 
translated to fast computations, that even if they were not accurate compared to the full model, they served as great candidates to
explore the different components of the computational model of the biosensor. We used this model to shine some light on the effects of multiple factors such as 
the orientation of the protein, the presence of charges in the protein, the number of proteins, and the effect of the incoming electric field \ref{sec:ell_study}.
We found that using real dielectrics for the protein and the embedding medium does not affect the shift in the resonance frequency, 
as long as they are averaged values from the real part of the original complex data. However, using real dielectrics does affect the 
intensity of the extinction cross section, which is a problem if the results are intended to be used for sensitivity calculations \ref{sec:diel_study}. 
These reduced-order models, despite not being accurate compared to the full protein model, provide fast computed insights that can be used 
in the exploratory stage on the design or study of biosensors.

We matched and replicated the findings of computational results in the general area of nanostructure responses to electromagnetic waves 
from Rockstuhl et al. \cite{rockstuhl2005} and Ellis et al.\cite{ellis2016} (section \ref{sec:rep_rockstuhl} and \ref{sec:rep_val_ellis}). We validated
\pygbe against experimental results of reflectance of localized surface phonon polariton (SPhP) nanostructures from Ellis et 
al. (section \ref{sec:rep_val_ellis}). Both replication and validation studies were difficult to achieve due to insufficient data from the experiments 
and/or simulation details, and we stated that the lack of reproducible open practices is the primary cause. We concluded that open reproducible 
practices are needed not only to guarantee the work published is possible to reproduce, but also for future replicability and validation studies. 

All of the results of this work, along with the materials needed to reproduce or replicate them, are publicly available in the form of 
reproducibility packages (sections \ref{sec:repro_lspr}, \ref{sec:repro_ell}, and \ref{sec:repro_val}). The nanoplasmonics extension of \pygbe, and 
the results of this work, can offer a valuable computational approach for future studies on the field, and aid in the design of LSPR biosensors 
as well as SPhP based sensors. 


