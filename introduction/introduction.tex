% !TEX root = ../thesis-sample.tex
\chapter{Introduction} \label{chap:intro}

A biosensor is a measuring device used to detect the presence or concentration of a biomolecule. The general system includes 
a probe with a sensitive biological receptor, a transducer and a detector. For the past three decades, research in biosensing 
has rapidly grown with applications in multiple areas like medical diagnoses, disease monitoring, drug delivery studies, 
and detection of biological agents \cite{Turner2000, Mohanty2006, Mehrotra2016}. Optical biosensors are a particular type of biosensor 
known for their sensitivity and real-time detection. Among many optical sensing techniques we have fluorescence, Raman spectroscopy, lithography, 
and surface plasmon resonance. Biosensors based on surface plasmon resonance (SPR) changes in thin gold films are some of the most 
commonly used optical biosensors.  These sensors exploit a particular type of electromagnetic waves called surface plasmon polaritons for the
detection of a biomolecule \cite{Homola2008}. However, when using metallic nanoparticles instead of thin films, a powerful resonance peak 
shows in the visible range, a phenomenon known as Localized Surface Plasmon Resonance (LSPR). When electromagnetic waves excite the free electrons on 
the surface of a conductive nanoparticle, they resonate with the incoming electric field forming plasmons. Most of the incoming energy is either 
absorbed by the nanoparticle, or scattered in multiple directions, creating a large shadow behind the scatterer (a.k.a extinction cross section). In LSPR, 
the resonance wavelength highly depends on the refraction index of the environment, the geometry of the nanoparticle and its material. LSPR biosensors measure 
the shift in the plasmon resonance wavelength ($\Delta\lambda$) of the metallic nanoparticle as a biomolecule approaches it. Biosensing based on LSPR has 
is more precise and provides high-sensitivity detection \cite{Sepulveda2009}. This technique has been used in multiple areas of biological detection, 
such as viruses and cancer \cite{Wang2010, Liu2014, Zhu2016}. 

The physics of localized surface plasmon resonance are modeled and solved by Maxwell's equations, and they are generally solved numerically 
by using finite difference time-domain (FDTD), boundary element, or finite element methods \cite{SolisTaboadaObelleiroLiz-MaarzanGarciadeabajo2014}.
These methods have been used to study the optical properties of dielectric or metallic nanoparticles \cite{Hohenester2018,HohenesterTrugler2012,
JungPedersenSondergaardPedersenLarsenNielsen2010, VideenSun2003,MayergoyzFredkinZhang2005, MayergoyzZhang2007}, interactions between nanoparticles
and electron beams \cite{GarciadeabajoAizpurua1997, GarciadeabajoHowie2002}, and surface plasmon resonance sensors. For the latter application, 
researchers use simple mathematical models for the interaction between the nanoparticle and the biomolecules, like representing the molecules as simple 
spheres \cite{DavisGomezVernon2010,AntosiewiczApellClaudioKall2011, SantiagoCordobaETal2011, ShenETal2013, UngerETal2009} or a prisms \cite{DanHu2014}, using 
constant real dielectrics for the analyte \cite{NghiemETal2012, SantiagoCordobaETal2011, ShenETal2013, UngerETal2009}, or representing the embedding medium 
and dissolved biomolecules with an effective permittivity \cite{JungCampbellChinowskyMarYee1998,WilletsVandyune2007,PhanETal2013}.

Experimental research drives progress in the field of biosensing, which can be costly and time consuming. We believe that 
computational approaches could complement the design process and play a key role in the optimization of biosensors. Using computational approaches
can not only reduce the cost of the exploratory stages, but also give access to details that are not always available in experimental settings. For example, 
the distance between the sensor and the biomolecule, the amount of analytes in the vicinity of the sensor and their orientation, and other factors such like
the inclusion of the charges in the protein and the effects of the magnitude in the electric field when they are present. There are studies that show that 
the sensitivity of the sensor depends highly on the distance between the nanoparticle and the analyte \cite{HaesETal2004}. Other experimental studies complement
their results with models that ignore the presence of the biomolecules. Beuwer et al.~\cite{BeuwervanHoofZijlstra2018} and Henkel et al.~\cite{HenkelETal2018} 
use a boundary element method (BEM) in studies of the sensitivity of plasmonic sensors relying on (at least) two metallic nanoparticles 
(one on the sensor and one attached to the analyte). There is experimental evidence that shows that LSPR sensors are sensitive enough to detect conformational
changes in the biomolecules used as analytes \cite{HallETal2011}. Oversimplified models will not capture such details. 

When the wavelength of the incoming field is much larger than the size of the nanoparticle ($\lambda>>d$), the physics of the optical phenomenon of 
localized surface plasmon resonance can be reduced to electrostatics. This approximation enables us to use our boundary integral 
electrostatic solver, \pygbe, to compute the extinction cross section of metallic or dielectric nanoparticles, and to study how the LSPR response varies in the 
presence of a biomolecule. We show how sensitivity is affected by the distance between the sensor and the analyte in LSPR biosensors. We exhibit the impact 
of using different geometrical models for the biomolecules on the LSPR response. We analyze the effect of including charges in the protein model. We present 
the differences of using real versus complex dielectric models for both the analyte and embedding medium.

\pygbe is a Python implementation of continuum electrostatic theory that relies on the boundary element method, its original application included 
the computing of solvation energy of biomolecular systems \cite{CooperBardhanBarba2013}, and the studying of protein orientation near charged 
nanosurfaces \cite{CooperClementiBarba2015}. In 2017, we extended \pygbe to the field of nanoplasmonics, by treating localized surface plasmon resonance 
(LSPR) quasistatically \cite{ClementiETal2017}. The boundary element solver in \pygbe is accelerated algorithmically via a treecode, an $\mathcal{O}(N\log N)$ 
fast-summation method, and on hardware by taking advantage of graphic processing units (GPUs). This allows us to handle problems in the order of half a million 
boundary elements, or more, thus being able to use an accurate representation of the biomolecular surface. There are other research softwares that could be use 
for this application, like BEM++ \cite{SmigajETal2015} and a Matlab toolbox called MNPBEM \cite{HohenesterTrugler2012}, which have the capability to solve the 
full Maxwell's equations and the electrostatic approximation in the long-wavelength limit. However, we believe that the size of the
problems these softwares can solve, in terms of the number of boundary elements,  may not be enough to resolve the details of the biomolecules.

 
To the best of our knowledge, \pygbe is the only open-source software that uses a fast algorithm ($\mathcal{O}(N\log N)$),
for $N$ unknowns—and hardware acceleration on GPUs to compute the extinction cross-sections of arbitrary geometries to explore
nanophotonics applications in the quasistatic limit. Our software is openly developed via its repository on 
Github (\url{https://github.com/barbagroup/pygbe}), has been verified (see section \ref{sec:verification}) against an analytical solution based 
on the Mie theory for silver nanospheres, validated against experimental results of reflectance of localized surface phonon polariton 
(SPhP) nanostructures from Ellis et al. (see section \ref{ssec:validation}), and we share it under the BSD 3-clause license.

This work follows careful reproducibility practices, and the materials necessary to reproduce or replicate our results are publicly available in 
reproducibility packages. Specific information about the reproducibility packages can be found at the end of each chapter with results. 
