% !TEX root = ../thesis-sample.tex

\chapter{LSPR response to Bovine Serum Albumin} \label{chap:lspr_response_bsa}
\graphicspath{{lspr_response_bsa/figs/}}

Localized surface plasmon resonance (LSPR) biosensors detect target molecules
by tracking frequency shifts in the plasmon resonance of metallic nanoparticles
in the presence of analytes \cite{WilletsVandyune2007}. The chapter presents the 
modeling of LSPR biosensors using \pygbe. We compute the extinction cross section 
of a silver nanosphere with bovine serum albumin (BSA) proteins (PDB code: 4FS5,
BSA dimmer) in different locations around it. 


{\color{red}
We might need to add something else here
}

\section{Grid convergence analysis} \label{sec:grid_conv_bsa}
We perform a grid convergence study to ensure that the meshes are correctly
resolving the numerical solutions. We performed the convergence analysis of
the system sketched in in Figure \ref{fig:analyte-sensor}. Given that we 
compute the extinction cross section by integrating over the sphere, we set 
a fixed mesh density for the protein and refined the mesh of the nanosphere 
(512, 2048, 8192, 32768 elements). For the protein, we found that a mesh with
two triangles per $\text{\AA}^2$ was fine enough for the convergence 
analysis, resulting in $N_{prot} = 98116$ elements.
We use the same conditions used in the grid convergence analysis of the 
isolated nanoparticle of section \ref{sub_sec:grid_conv_iso}, presented
in Tables \ref{table:quadparams1} and \ref{table:treeparams1}. The protein 
dielectric constant for a wavelength of 380 nm is $2.7514 + 0.2860i$, this  
value was computed using a functional relationship provided by Phan
 et al.~\cite{PhanETal2013}. The protein was located at a distance of 
 $d=$ 1 nm of the sphere along the z-axis, such that its dipole moment 
 was aligned with the y-axis. We show the errors in Figure  \ref{fig:err_sph-bsa} 
 and table \ref{table:err_sph-bsa} which were computed using the Richardson extrapolated
{\color{red} (see Section, cite section that explains rich extra)} value of 
the extinction cross section $C_{ext}= 1778.73$ nm$^2$.

\begin{figure}%[h] %  figure placement: here, top, bottom, or page
    \centering
    \includegraphics[width=0.65\textwidth]{convergence_bsa_sensor_R8_d1_w380.pdf} 
    \caption{Grid-convergence study of extinction cross-section of a spherical silver
             nanoparticle with a BSA protein at $d=1$ nm. 
             Figure, plotting script and auxiliary files available 
             under \textsc{cc-by} \cite{ClementiETal2018c}.}
    \label{fig:err_sph-bsa}
 \end{figure}

 \begin{table}%[h]
    \centering
    \caption{\label{table:err_sph-bsa} Estimated percentage error of the BSA-sensor 
    system (Fig.~\ref{fig:analyte-sensor}), with respect to the extrapolated value 
    (using Richardson extrapolation).} 
    \begin{tabular}{c c}
    \hline%\toprule
    N & \% error \\
    \hline%\midrule
     $512$ & $29.39$ \\
     $2048$ & $7.13$ \\
     $8192$ & $1.82$ \\
     $32768$ & $0.46$ \\
    \hline%\bottomrule
    \end{tabular}
\end{table}

The observed order of convergence is 0.99, and we can see in Figure
\ref{fig:err_sph-bsa} that the error decays with the number of boundary elements
at a rate of $1/N$, which is consistent with the verifications results showed
in Section \ref{sec:verification}. This shows that the numerical solutions computed
with \pygbe are correctly resolved by the meshes.

\section{Plasmon resonance frequency shifts} \label{sec:shift_bsa}

When a target molecule approaches the metallic nanoparticle, the resonance frequency
of this nanoparticle shifts. In this section, we computed the LSPR response as a
function of the wavelength in the presence of BSA protein. We optimized run times 
without compromising accuracy by using a relaxed set of parameters. For the protein
mesh density we used one element per $\text{\AA}^2$ ($N_{prot} = 45140$) and for the
sphere mesh we used $N_{sensor} = 32768$ elements. These calculations used the same
parameters from Tables \ref{table:quadparams2} and \ref{table:treeparams2}, which 
resulted in a percentage error below $1\%$, with respect to the Richardson-extrapolated
value. The run time for one frequency when two proteins are present, is approximately 
$15$ min using a NVIDIA Tesla K40c GPU.



\begin{center}
    \begin{figure} %  figure placement: here, top, bottom, or page
       \centering
       \includegraphics[width=0.65\textwidth]{2prot_1nm_z_R8nm.pdf} 
       \caption{Sensor protein display: BSA located at $\pm 1$ nm of the 
                nanoparticle in the $z$-direction. Figure, plotting script and auxiliary 
                files available under \textsc{cc-by} \cite{ClementiETal2018e}.}
       \label{fig:display_z}
    \end{figure}
    \end{center}

Figure \ref{fig:display_z} shows a visualization of the meshes setup for these 
calculations, with two BSA proteins located at $d=1$ nm away from the spherical 
silver nanoparticle, along the $z$ axis. The position of the BSA molecule at $+z$ 
axis was the same used for the convergence analysis in section \ref{sec:grid_conv_bsa},
while the protein located in the $-z$ position is a 180$^\circ$ solid rotation 
about the $y$ axis, of the BSA in $+z$.  Figure \ref{fig:2pz_response} shows the results
of the calculations between 382 and 387 nm every $0.25$ nm, near the peak seen in Figure
\ref{fig:verif_sph}. In Figure \ref{fig:2pz_response} we have the variation of the 
extinction cross section with respect to wavelength for the isolated nanoparticle 
($d=\infty$) and with BSA proteins located at $d=1$ nm apart from the nanosphere. The
result shows a redshift ($0.5$ nm) in the resonance frequency due to the presence of 
the BSA analytes.

\begin{figure} %[h] %  figure placement: here, top, bottom, or page
    \centering
    \includegraphics[width=0.65\textwidth]{2pz_R8nm.pdf} 
    \caption{Extinction cross-section as a function of wavelength for an $8$ nm
             silver sphere immersed in water with two BSA proteins placed 
             $\pm 1$ nm away from the surface in the $z$-direction, and at
             infinity (no protein).}
    \label{fig:2pz_response}
 \end{figure}

 We also study the effect of the location of the proteins, we performed the same 
 calculations but now placing the BSA analytes along the $x$ and $y$ axis at $\pm 1$ nm,
 as shown in Figure \ref{fig:display_xy}. We obtain these configurations by performing
 a 90$^\circ$ solid rotation of the $z$-configuration (Figure \ref{fig:display_z})
 along the $x$- and $y$-axis, respectively. Figure \ref{fig:2pxy_response} shows the results for each 
 configuration.


 \begin{figure}
    \centering
    \subfloat{\includegraphics[width=0.65\textwidth]{2prot_1nm_x_R8nm.pdf}} \\
    \subfloat{\includegraphics[width=0.65\textwidth]{2prot_1nm_y_R8nm.pdf}} 
     \caption{Sensor protein display: BSA located at $\pm 1$ nm of the nanoparticle in the
            $x$-direction (top) and $y$-direction (bottom). Figure, plotting script and auxiliary files available under \textsc{cc-by} \cite{ClementiETal2018e}.}
     \label{fig:display_xy}
 \end{figure}

 \begin{figure}%[t] %  figure placement: here, top, bottom, or page
    \centering
    \subfloat{\includegraphics[width=0.65\textwidth]{{2px_R8nm}.pdf}}\\
    \subfloat{\includegraphics[width=0.65\textwidth]{{2py_R8nm}.pdf}} 
    \caption{Extinction cross-section as a function of wavelength for an $8$-nm
             silver sphere immersed in water with two BSA proteins placed at
             $\pm 1 $ nm away from the surface in the $x$-direction (top) and
              $y$-direction (bottom), and at infinity (no protein).}
    \label{fig:2pxy_response}
 \end{figure}


\section{Sensitivity study} \label{sec:sensitivity}
