% !TEX root = ../thesis-sample.tex

\chapter{LSPR response to Bovine Serum Albumin} \label{chap:lspr_response_bsa}
\graphicspath{{lspr_response_bsa/figs/}}

Localized surface plasmon resonance (LSPR) biosensors detect target molecules
by tracking frequency shifts in the plasmon resonance of metallic nanoparticles
in the presence of analytes \cite{WilletsVandyune2007}. The chapter presents the 
modeling of LSPR biosensors using \pygbe. We compute the extinction cross section 
of a silver nanosphere with bovine serum albumin (BSA) proteins (PDB code: 4FS5,
BSA dimmer) in different locations around it. 


{\color{red}
We might need to add something else here
}

\section{Grid convergence analysis} \label{sec:grid_conv_bsa}
We perform a grid convergence study to ensure that the meshes are correctly
resolving the numerical solutions. We performed the convergence analysis of
the system sketched in in Figure \ref{fig:analyte-sensor}. Given that we 
compute the extinction cross section by integrating over the sphere, we set 
a fixed mesh density for the protein and refined the mesh of the nanosphere 
(512, 2048, 8192, 32768 elements). For the protein, we found that a mesh with
two triangles per $\text{\AA}^2$ was fine enough for the convergence 
analysis, resulting in $N_{prot} = 98116$ elements.

\section{Plasmon resonance frequency shifts} \label{sec:shift_bsa}

\section{Sensitivity study} \label{sec:sensitivity}
