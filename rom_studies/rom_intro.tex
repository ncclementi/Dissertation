% !TEX root = ../thesis_main.tex

\chapter{Reduced order model for protein representation}

In the literature there are several reports that use Bovine Serum Albumin as an analyte
to simulate the behavior of a bioconjugate sensor. However, they use a simplified model
of the protein. For example, a layer of protein-water solution \cite{PhanETal2013}, 
or just a protein dielectric layer \cite{NghiemETal2012}, others model the
protein as a triangular prism \cite{DanHu2014}, or even a small sphere with a 
constant dielectric \cite{SantiagoCordobaETal2011, UngerETal2009}. Phan and 
coworkers \cite{PhanETal2013} present a Lorentz-Drude model for the complex 
dielectric of the BSA which we use in our work, while most of the literature 
use a constant dielectric with no losses to represent the BSA 
(refractive index of $n= 1.9$ \cite{NghiemETal2012}, $n= 1.45$ \cite{SantiagoCordobaETal2011}) or
other proteins used as analytes ($n=1.58$ polystrene \cite{UngerETal2009}, $n=1.57$ 
streptavidin \cite{ShenETal2013}). 

Research shows that the shape of both monomer and dimmer of 
BSA are prolate ellipsoids \cite{MoserETal1966, SquireETal1968, WrightETal1975}. However,
this is information, to the best of our knwoledge, hasn't been used to model the 
protein in computational approaches. Our BSA model, based on its crystal 
structure, can be consider as a good reference of the actual shape of the protein. We 
explore the effects of using reduced-order models, like ellipsoids, by comparing 
the results with the full model, and study the consequences of these 
simplifications. We study the effect of the shape and volume of the protein
(see section CITE SECT) as well as the presence of charges in it. 

To the best of our knowledge, there is no model in the literature that counts 
for the charges inside the protein. We study the effect on the LSPR response 
in the presence and omission of charges and how this varies depending on the 
intensity of the incoming electric field (see section CITE SECT).\\  