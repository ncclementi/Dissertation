% !TEX root = ../thesis-sample.tex

% --------- FRONT MATTER PAGES ---------------------
% Title of the thesis
\title{COMPUTATIONAL NANOPLASMONICS FOR BIOSENSING APPLICATIONS: A BOUNDARY INTEGRAL IMPLEMENTATION IN THE QUASISTATIC LIMIT.}

% Author name
\author{Natalia Carolina Clementi}

% Previous degrees
\bsdepartment{Physics}
\bsschool{Universidad Nacional de C{\'o}rdoba.}
\bsgrad{2013}

\msdepartment{MS department}
\msschool{University}
\msgrad{year}
%\showmsdegree % you can show or hide the MS degree line 
\hidemsdegree

% PhD degree commands
% Committee
\showcommitteepage % hide this page if you're doing a MS thesis
%\hidecommitteepage 
\committee{ %
Lorena A. Barba, Professor of Mechanical and Aerospace Engineering,\\ 
Dissertation Director\\ % remember to add a space between committee members

Sanyia LeBlanc, Associate Professor of Mechanical and Aerospace Engineering,
Committee Member\\

Kausik Sarkar, Professor of Mechanical and Aerospace Engineering, \\
Committee Member\\

Yongsheng Leng, Professor of Mechanical and Aerospace Engineering, \\
Committee Member\\

Christopher Cooper Villagr{\'a}n, Assistant Professor of Mechanical Engineering,
Committee Member\\

}

% Chair must be entered separately for formatting reasons.
\chair{Lorena A. Barba}
\chairtitle{Professor of Mechanical and Aerospace Engineering}
% Department
\department{Mechanical and Aerospace Engineering}

\phdgrad{May 16, 2021}
\defensedate{February 16, 2021}
% Year of completion for copyright page and perhaps other places
\year=2021

% Copyright page
%\copyrightholder{Someone else}

% Dedication
\dedication{To all the women scientists to come. %
}

% Acknowledgments
\acknowledgments{
PENDING
}

% -----------------------------------------------------------------
% Typically only one of Preface/Foreward/Prologue would be in your thesis.
% To choose one simply delete the others and they will automatically dissappear

% Preface
\preface{
    This is the preface. 
    It's another front matter page that offers additional detail into your work.
    Typically, only one (preface OR prologue OR foreword) is used. 
    You can remove the other sections by deleting them inside \texttt{tex/frontmatter.tex} or using the appropriate show or hide commands.
}

\prologue{
    This is the prologe. 
    It's another front matter page that offers additional detail into your work.
    Typically, only one (preface OR prologue OR foreword) is used. 
    You can remove the other sections by deleting them inside \texttt{tex/frontmatter.tex} or using the appropriate show or hide commands.
}

\foreword[2]{
    This is the forword. 
    It's another front matter page that offers additional detail into your work.
    Typically, only one (preface OR prologue OR foreword) is used. 
    You can remove the other sections by deleting them inside \texttt{tex/frontmatter.tex} or using the appropriate show or hide commands.
}
% ----------------------------------------------------------------------

% commands to show or hide front matter pages

\showcopyright
\showabstract
\showcommitteepage
\showdedication
\showacknowledgments
\hidepreface
\hideprologue
\hideforeword

% ------------ TABLE OF CONTENTS ----------------------
% Commands to hide or show lists of figures, tables, etc.
\showlistoffigures
\showlistoftables
\hidenomenclature

\makeglossaries

% Some abstract text
\abstract{
Localized surface plasmon resonance biosensors provide high sensitivity in detecting biomolecules via shifts in the resonance 
frequency when targets are in the vicinity. The physics of this field are modeled with Maxwell's equations but in the long-wavelength 
limit, electrostatics serves as a good approximation. This work uses this approach, expanding the open-source \pygbe software
to compute the extinction cross section of metallic and dielectric nanoparticles in the presence of biomolecules. \pygbe is a Python
boundary integral research software for continuum electrostatics, where the computationally expensive parts are accelerated on GPU 
hardware. It is also algorithmically accelerated via a treecode that offers $\mathcal{O}(N \log N)$ computational complexity, that 
allows \pygbe to handle problems with over half a million boundary elements. These features enable \pygbe to represent the target 
molecule as a solvent excluded mesh based on the crystal structure and capture its complexity.

We demonstrate the suitability of \pygbe for computational nanoplasmonics for biosensing applications. We verify our solver 
against an analytical solution of the extinction cross section as a function of the wavelength for a silver nanosphere in a
water medium. We present a sensitivity study for a biosensor model, where we compute the resonance frequency shift for different
distances between the proteins and the nanoparticle, and show that the shift decays as the proteins move away from the sensor.
Computational studies in the field of nanoplasmonic biosensing have used full Maxwell's equations with simplified models 
of the sensor-analyte system. We show that reduced order models for the analyte are not accurate compared to the full 
protein representation extracted from its crystal structure. However, using them results in faster computations that
we use to explore the multiple factors involved in the computational model of a biosensor. 
We replicate two studies in electromagnetic excitations on silicon carbide nanostructures, where the quantity of interest 
is the wavelength of the resonance peak. Finally, we validate \pygbe by comparing against experiments that measured 
the polarized reflectance of silicon carbide nanopillars. All the results in this work are reproducible, and all the materials 
needed to run the computations and re-create the figures are openly available in the form of reproducibility packages
(see sections \ref{sec:repro_lspr}, \ref{sec:repro_ell}, and \ref{sec:repro_val}).
}
