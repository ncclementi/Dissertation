% !TEX root = ../thesis-sample.tex

% --------- FRONT MATTER PAGES ---------------------
% Title of the thesis
\title{Computational Nanoplasmonics for Biosensing Applications: A Boundary Integral Implementation in the Quasistatic Limit.}

% Author name
\author{Natalia Carolina Clementi}

% Previous degrees
\bsdepartment{Physics}
\bsschool{Universidad Nacional de C{\'o}rdoba.}
\bsgrad{2013}

\msdepartment{MS department}
\msschool{University}
\msgrad{year}
%\showmsdegree % you can show or hide the MS degree line 
\hidemsdegree

% PhD degree commands
% Committee
\showcommitteepage % hide this page if you're doing a MS thesis
%\hidecommitteepage 
\committee{ %
Lorena A. Barba, Professor of Mechanical and Aerospace Engineering,\\ 
Dissertation Director\\ % remember to add a space between committee members

Sanyia LeBlanc, Associate Professor of Mechanical and Aerospace Engineering,
Committee Member\\

Kausik Sarkar, Professor of Mechanical and Aerospace Engineering, \\
Committee Member\\

Yongsheng Leng, Professor of Mechanical and Aerospace Engineering, \\
Committee Member\\

Christopher Cooper Villagr{\'a}n, Assistant Professor of Mechanical Engineering,
Committee Member\\

}

% Chair must be entered separately for formatting reasons.
\chair{Lorena A. Barba}
\chairtitle{Professor of Mechanical and Aerospace Engineering}
% Department
\department{Mechanical and Aerospace Engineering}

\phdgrad{May 16, 2021}
\defensedate{February 16, 2021}
% Year of completion for copyright page and perhaps other places
\year=2021

% Copyright page
%\copyrightholder{Someone else}

% Dedication
\dedication{To all the women scientists to come. %
}

% Acknowledgments
\acknowledgments{
    No PhD student can survive without the support of their community. Along the years that 
    I spent in my PhD I was  fortunate to have a community that supported me both in the good and 
    rough moments. Being an international student meant I was away from my family and friends in 
    Argentina. Fortunately, I was able to make new friends that made me feel like home. I will 
    be forever be grateful for their support and encouragement during this period of my life.
    
    To start, I would like to thank my advisor Professor Lorena Barba, for giving me the opportunity  
    to fulfill my PhD in the United States, and support me throughout all these years. I am thankful 
    for her advice and teachings, and I know I will carry them for the rest of my career. She provided
    me opportunities to  meet incredible people from the open source community and exposed me to a 
    world outside of academia. 
    
    I would like to give special thanks to my thesis committee: Dr. Sanyia LeBlanc, Dr. Kausik Sarkar, 
    Dr. Yongsheng Leng, and Dr. Christopher Cooper, for dedicating their time to serve on my committee
     and for the insights and comments on my work. In particular, I am deeply grateful for the support 
     and advice of  Dr. Christopher Cooper that has been key to my progress over the years. 
    
    While at George Washington University, I spent countless hours in the office with my lab mates. I 
    would like to thank Oliver Mesnard, Tingyu Wang, Pi-Yueh Chuang and Gil Forsyth for all the time 
    and discussions I shared with them, and most recently to our newest members, Anastasiia Sarmakeeva 
    and Paulina Rodriguez. At GWU, I made friends from other labs and I'm thankful for all the moments
     and beers shared. I would like to particularly thank Jenna Osborn, Abhilash Malipeddi and Meg 
     Anderson for their support and company during the last couple of years. All the lunches, coffees, 
     beers and talks have been very helpful, especially in times that I needed to be brought back to Earth 
    
    When I arrived in the United States, I was lucky to meet the Vanella family. Marcos and Patri opened their
    house and heart to me, and they have been my Argentinian family in the US. I will be forever grateful for
    having them in my life. I would also like to thank Any Gomez Vidal, for the extensive talks about the
    journey of doing a PhD and for always being there when it was becoming overwhelming.
    
    I want to express my gratitude to my family, who have always supported my choices and my career. I want 
    to thank my parents Susana Priotto and Eduardo Clementi, for encouraging me to choose my own path and 
    stood by my decisions. I would also like to thank the Hea family, in particular to Bob and Rima, who 
    have always expressed their support and encouragement throughout these years. 
    
    Last by not least, I would like to give a special thanks to my partner Will Hea. He has been by my side
    for most of my PhD journey, and has given me unconditional love and support. Thanks for always believing 
    in me, and for always pushing me forward. Once he told me "it doesn't count until it's done", now I can 
    say I'm Phinally Done, and thanks for hanging in there with me. Will, I'm grateful to be able to share 
    this accomplishment with you, and I look forward to sharing the new chapters to come.     
}

% -----------------------------------------------------------------
% Typically only one of Preface/Foreward/Prologue would be in your thesis.
% To choose one simply delete the others and they will automatically dissappear

% Preface
\preface{
    This is the preface. 
    It's another front matter page that offers additional detail into your work.
    Typically, only one (preface OR prologue OR foreword) is used. 
    You can remove the other sections by deleting them inside \texttt{tex/frontmatter.tex} or using the appropriate show or hide commands.
}

\prologue{
    This is the prologe. 
    It's another front matter page that offers additional detail into your work.
    Typically, only one (preface OR prologue OR foreword) is used. 
    You can remove the other sections by deleting them inside \texttt{tex/frontmatter.tex} or using the appropriate show or hide commands.
}

\foreword[2]{
    This is the forword. 
    It's another front matter page that offers additional detail into your work.
    Typically, only one (preface OR prologue OR foreword) is used. 
    You can remove the other sections by deleting them inside \texttt{tex/frontmatter.tex} or using the appropriate show or hide commands.
}
% ----------------------------------------------------------------------

% commands to show or hide front matter pages

\showcopyright
\showabstract
\showcommitteepage
\showdedication
\showacknowledgments
\hidepreface
\hideprologue
\hideforeword

% ------------ TABLE OF CONTENTS ----------------------
% Commands to hide or show lists of figures, tables, etc.
\showlistoffigures
\showlistoftables
\hidenomenclature

\makeglossaries

% Some abstract text
\abstract{
Localized surface plasmon resonance biosensors provide high sensitivity in detecting biomolecules via shifts in the resonance 
frequency when targets are in the vicinity. The physics of this phenomenon is modeled with Maxwell's equations, but in the long-wavelength 
limit, electrostatics serves as a good approximation. This work uses this approach, expanding the open-source \pygbe software
to compute the extinction cross section of metallic and dielectric nanoparticles in the presence of biomolecules. \pygbe is a Python
boundary integral research software for continuum electrostatics, where the computationally expensive parts are accelerated on GPU 
hardware. It is also algorithmically accelerated via a treecode that offers $\mathcal{O}(N \log N)$ computational complexity, allowing
\pygbe to handle problems with over half a million boundary elements. These features enable \pygbe to represent the target 
molecule as a solvent excluded mesh based on the crystal structure and capture its complexity. Our results show grid convergence
as 1/N, and accurate computation of the extinction cross-section as a function of wavelength. The computations are compared to 
the analytical solution for the case of an isolated nanoparticle, and compared to the Richardson extrapolation when we have 
the presence of an analyte.

We demonstrate the suitability of \pygbe for computational nanoplasmonics for biosensing applications. We verified our solver 
against an analytical solution of the extinction cross section as a function of the wavelength for a silver nanosphere in a
water medium. We present a sensitivity study for a biosensor model, where we computed the resonance frequency shift for different
distances between two proteins located in the z-axis and the nanoparticle, and show that the shift decays from $0.75$ nm
to $0.25$ nm as the proteins move away from the sensor from a distance of $0.5$ nm to $2$ nm. We also show that this behavior 
varies depending on the position of the proteins as we see no shift when the proteins are located at $1$ nm on the $x$ or $y$
axis.

Computational studies to date in the field of nanoplasmonic biosensing have used full Maxwell's equations with simplified models 
of the sensor-analyte system. We show that reduced order models for the analyte are not sufficiently accurate compared to the full 
protein representation extracted from its crystal structure. Compared to the crystal structure model, the volume-equivalent model
overestimates the shift by $50$ $\%$ and the surface-equivalent by $400$ $\%$. Despite the overestimation of the shift, we show 
that ellipsoidal models capture orientation effects, while spherical models do not. We show that using the volume-equivalent is 
still valuable since it results in faster computations that we use to explore the multiple factors involved in the computational
model of a biosensor. 

We replicated two studies in electromagnetic excitations on silicon carbide nanostructures, where the quantity of interest 
is the wavelength of the resonance peak. Despite the differences in our method, we replicated a result by Rockstuhl et al. \cite{rockstuhl2005}
where they used a two-dimensional boundary element method on silicon carbide rectangular cylinders. The second replication case corresponds to 
a result in the work of Ellis et al. \cite{ellis2016}, where they looked at the aspect ratio effects on high-order modes of
localized surface phonon-polariton nanostructures. We partially replicated the results since the wavenumber position of certain modes match
while for others there is a discrepancy that cannot be explained without detailed information (not provided) of the original simulations. Finally, we
validated \pygbe by comparing against experiments from the work of Ellis et al. \cite{ellis2016} that measured 
the polarized reflectance of silicon carbide nanopillars. We perform a first order correction in our results, which leads to a match on the wavenumber for the 
dominant mode and two more, but with some differences in other minor modes.

All the results in this work are reproducible, and all the materials 
needed to run the computations and re-create the figures are openly available in the form of reproducibility packages 
that include input files, scripts to run the simulations and process the results, and any additional data needed 
to reproduce the results.}
